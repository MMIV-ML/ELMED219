\documentclass[11pt,a4paper]{article}
\usepackage[utf8]{inputenc}
\usepackage{times}
\usepackage{geometry}
\geometry{a4paper}

\title{Application to the Regional Committee for Medical and Health Research Ethics (REK)}
\author{Dr. [Your Name]}
\date{November 27, 2023}

\begin{document}

\maketitle

\section{Introduction}
\subsection{Study Title}
Deciphering the Interplay Between Gut Microbiota and Brain-Gut Axis in Irritable Bowel Syndrome: A Longitudinal Study

\subsection{Principal Investigator}
Dr. [Your Name], [Your Department], [Your Institution]

\subsection{Contact Information}
\noindent [Your Address] \\
\noindent [Your Email] \\
\noindent [Your Phone Number]

\section{Study Purpose}
This study aims to explore the complex interactions between gut microbiota and the brain-gut axis in patients with Irritable Bowel Syndrome (IBS). The primary objective is to identify specific microbiota changes associated with IBS and understand how these alterations influence brain-gut interactions and symptom severity. The research seeks to bridge a critical knowledge gap in the pathophysiology of IBS, contributing to the development of more effective diagnostic and therapeutic strategies. By elucidating the underlying mechanisms of IBS, the study holds the potential to significantly enhance patient care and improve quality of life for those affected by this condition.

\section{Study Design}
This is a longitudinal, observational study spanning 18 months, involving 200 participants - 100 diagnosed with IBS and 100 healthy controls. The study will employ a comprehensive approach, including microbiota profiling through bi-monthly stool samples, brain imaging via functional MRI at the start and end of the study, and regular psychological assessments to track symptom progression and mental well-being. The design allows for a detailed examination of the temporal relationships between microbiota composition, brain-gut axis function, and clinical manifestations of IBS. This approach is crucial for identifying potential biomarkers and understanding the dynamic nature of IBS.

\section{Ethical Considerations}
\subsection{Informed Consent}
Informed consent will be obtained from all participants, explaining the study's purpose, procedures, and their rights.

\subsection{Confidentiality}
Participants' data will be stored in a secure, encrypted database and will be de-identified to maintain confidentiality.

\subsection{Data Protection}
We will adhere to GDPR and other relevant data protection regulations for handling and storing sensitive personal data.

\section{Potential Risks and Benefits}
The study poses minimal risk to participants, primarily related to the discomfort of stool sample collection and MRI scanning. While there is no direct benefit to participants, the study contributes to a greater understanding of IBS.

\section{Funding and Conflict of Interest}
The study is funded by [Funding Source]. There are no conflicts of interest to declare.

\section{Conclusion}
This study will provide valuable insights into the complex interplay between gut microbiota and the brain-gut axis in IBS, potentially leading to more targeted treatment strategies.

\section{Attachments}
1. Informed Consent Form\\
2. Participant Information Sheet\\
3. Study Protocol

\end{document}


